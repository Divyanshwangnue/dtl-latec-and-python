\documentclass[10pt,a4paper]{report}

\usepackage[utf8]{inputenc}
\usepackage{amsmath}
\usepackage{amsfonts}
\usepackage{amssymb}
\usepackage{graphicx}
\usepackage[left=2cm, right=2cm, top=2cm, bottom=2cm]{geometry}

\author{Divyansh Wangnue 112103037 div1 S3}
\title{Bibliography}
\begin{document}
\maketitle


\section{System Programming in C}

C is a Low Level Programming Language used to design Operating systems, Data Structures and algorithms. The C library provides a handful of functions for translating an errno value to the
corresponding textual representation. This is needed only for error reporting, and the like;
checking and handling errors can be done using the preprocessor defines and errno directly.
Refer \cite{gfg} for more information about file handling in C...
This function prints to stderr (standard error) the string representation of the current error
described by errno, prefixed by the string pointed at by str, followed by a colon. To be useful,
References at  \cite{C}..
the name of the function that failed should be included in the string. For example:\\


\includegraphics[scale=1]{"OurCommandLineUtil.png"}
\\ [8pt]
This code  was written by \cite{atharva}. We believe that the overall propensity for this to happen is told well by \cite{somesite}
The former function returns a pointer to a string describing the error given by errnum. The
string may not be modified by the application, but can be modified by subsequent perror( )
and strerror( ) calls. In this manner, it is not thread-safe.

For a few functions, the entire range of the return type is a legal return value. In those cases,
errno must be zeroed before invocation, and checked afterward (these functions promise to
only return a nonzero errno on actual error).
\\ [12pt]
 system("gnome-terminal")  is a way to spawn a new Linux terminal ... CREDITS given to\cite{gfg}

\begin{thebibliography} {}
\bibitem{somesite} https://igm.univ-mlv.fr/~yahya/progsys/linux.pdf
\bibitem{gfg} Geeks For Geeks- File Handling in C
\bibitem{C}Dennis and Ritchie, C programming for Beginners

\end{thebibliography} 

\end{document}